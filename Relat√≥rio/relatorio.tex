\documentclass[10pt, conference, compsocconf]{IEEEtran}
\usepackage[utf8]{inputenc}
\usepackage[brazil]{babel}
%\usepackage{hyperref}
%\hypersetup{colorlinks=false}

% *** CITATION PACKAGES ***
%
%\usepackage{cite}
% *** GRAPHICS RELATED PACKAGES ***
%
\ifCLASSINFOpdf
  % \usepackage[pdftex]{graphicx}
  % declare the path(s) where your graphic files are
  % \graphicspath{{../pdf/}{../jpeg/}}
  % and their extensions so you won't have to specify these with
  % every instance of \includegraphics
  % \DeclareGraphicsExtensions{.pdf,.jpeg,.png}
\else
  % or other class option (dvipsone, dvipdf, if not using dvips). graphicx
  % will default to the driver specified in the system graphics.cfg if no
  % driver is specified.
  % \usepackage[dvips]{graphicx}
  % declare the path(s) where your graphic files are
  % \graphicspath{{../eps/}}
  % and their extensions so you won't have to specify these with
  % every instance of \includegraphics
  % \DeclareGraphicsExtensions{.eps}
\fi
% *** MATH PACKAGES ***
%
\usepackage[cmex10]{amsmath}

% *** PDF, URL AND HYPERLINK PACKAGES ***
%
\usepackage{url}

% correct bad hyphenation here
\hyphenation{op-tical net-works semi-conduc-tor}


\begin{document}
%
% paper title
% can use linebreaks \\ within to get better formatting as desired
\title{Predição de Renda Anual}


% author names and affiliations
% use a multiple column layout for up to two different
% affiliations

\author{\IEEEauthorblockN{Leandro Luciani Tavares, Luiz Benedito Aidar Gavioli, Victor Narcizo de Oliveira Neto}
\IEEEauthorblockA{Departamento de Computação (DComp)\\
Universidade Federal de São Carlos (UFSCar) \\
18052-780, Sorocaba, São Paulo, Brasil\\
leandro.ltavares@gmail.com, luizbag@gmail.com, vnarcizo@gmail.com}
}

% make the title area
\maketitle

\begin{abstract}
Resumo, deixar para o final.

\end{abstract}

\begin{IEEEkeywords}
component; formatting; style; styling;

\end{IEEEkeywords}

\section{Introdução}
Aprendizado de máquina é atualmente um dos principais campos da computação, sendo um sub-campo da Inteligência Artificial, o qual pretende dar habilidade às máquinas de obter conhecimento e se aperfeiçoarem em determinada tarefa, sendo ela, muitas vezes, pouco trivial ou até mesmo impossível para um humano realizar devido à complexidade ou ao volume dos dados.

Nesse projeto, essa tarefa consiste em comparar o desempenho dos principais métodos de classificação estudados na disciplina de Aprendizado de Máquina: o KNN(K-vizinhos mais próximos), a Regressão Logística, as Redes Neurais Artificiais (RNA), as Máquinas de Vetores de Suporte (SVM) e o Naive-Bayes, na classificação de padrões de renda \cite{trabalho}.

A predição de padrões de renda, é uma necessidade crescente de instituições fincanceiras, como bancos, seguradoras, factories, casas de câmbio, cooperativas de crédito, entre outras. Equipadas com ferramentas e dados, as instituições tem a possibilidade de fornecer serviços personalizados para seus clientes como, por exemplo, taxas diferenciadas para clientes baseados em suas rendas anuais. Ou adequar seu modelo de negócios a determinado tipo de consumidor, sabendo-se que, por exemplo, este tem um perfil inadimplente \cite{importance}.

Em análises econômicas, prever e classificar padrões de renda é parte fundamental, dada a necessidade de estimar o desenvolvimento econômico de um país, e traçar perfis dos cidadãos, como por exemplo, qual setor da economia tem os melhores salários, qual idade tem a parcela da população que possui maior renda anual. Além de auxiliar o planejamento econômico, controle de inflação e definição de taxas de juros \cite{importance2}.

A comparação se baseia na classificação de renda dos cidadãos norte-americanos, em 2 classes: os que possuem renda inferior à 50 mil dólares anuais ou os que possuem renda superior ou igual à 50 mil dólares anuais, com base em 14 atributos. A base utilizada para comparação pode ser consultada em \cite{base} e \cite{base2}.

\section{Pré-processamento}
A base de dados fornecida estava inicialmente separada em 2 arquivos, adult\_test e adult\_data, aos quais adicionou-se uma linha de cabeçalho para importação no Matlab, unificou-se ambos arquivos para o pré-processamento. A base de dados é composta por 14 atributos e 1 atributo-alvo, que representa se a renda é menor que 50 mil dólares anuais ou maior ou igual a 50 mil dólares anuais, sendo eles:
\begin{description}
\item[Age] \hfill \\ Atributo contínuo que representa idade;
\item[Workclass] \hfill \\ Atributo categórico que representa uma das 9 classes de trabalho;
\item[Fnlwgt] \hfill \\ Atributo contínuo;
\item[Education] \hfill \\ Atributo categórico que representa um dos 16 graus de escolaridade;
\item[Education-num] \hfill \\ Atributo continuo relacionado ao grau de escolaridade;
\item[Marital-status] \hfill \\ Atributo categórico que representa um dos 7 estados civis;
\item[Occupation] \hfill \\ Atributo categórico que representa uma das 14 áreas de trabalho;
\item[Relationship] \hfill \\ Atributo categórico que representa um dos 6 parentescos;
\item[Race] \hfill \\ Atributo categórico que representa uma das 5 etnias;
\item[Sex] \hfill \\ Atributo categórico que representa um dos 2 sexos possíveis;
\item[Capital-gain] \hfill \\ Atributo contínuo que representa o ganho de capital;
\item[Capital-loss] \hfill \\ Atributo contínuo que representa a perda de capital;
\item[Hours-per-week] \hfill \\ Atributo continuo que representa as horas trabalhadas por semana;
\item[Native-country] \hfill \\ Atributo categórico que representa um das 41 nacionalidades.
\end{description}

Após o carregamento removeu-se as amostras duplicadas, resultado em um total de 48813 amostras únicas, removeu-se também amostras com atributos idênticos porem com atributos-alvos distintos, resultando em 48785 amostras.


% An example of a floating figure using the graphicx package.
% Note that \label must occur AFTER (or within) \caption.
% For figures, \caption should occur after the \includegraphics.
% Note that IEEEtran v1.7 and later has special internal code that
% is designed to preserve the operation of \label within \caption
% even when the captionsoff option is in effect. However, because
% of issues like this, it may be the safest practice to put all your
% \label just after \caption rather than within \caption{}.
%
% Reminder: the "draftcls" or "draftclsnofoot", not "draft", class
% option should be used if it is desired that the figures are to be
% displayed while in draft mode.
%
%\begin{figure}[!t]
%\centering
%\includegraphics[width=2.5in]{myfigure}
% where an .eps filename suffix will be assumed under latex, 
% and a .pdf suffix will be assumed for pdflatex; or what has been declared
% via \DeclareGraphicsExtensions.
%\caption{Simulation Results}
%\label{fig_sim}
%\end{figure}

% Note that IEEE typically puts floats only at the top, even when this
% results in a large percentage of a column being occupied by floats.


% An example of a double column floating figure using two subfigures.
% (The subfig.sty package must be loaded for this to work.)
% The subfigure \label commands are set within each subfloat command, the
% \label for the overall figure must come after \caption.
% \hfil must be used as a separator to get equal spacing.
% The subfigure.sty package works much the same way, except \subfigure is
% used instead of \subfloat.
%
%\begin{figure*}[!t]
%\centerline{\subfloat[Case I]\includegraphics[width=2.5in]{subfigcase1}%
%\label{fig_first_case}}
%\hfil
%\subfloat[Case II]{\includegraphics[width=2.5in]{subfigcase2}%
%\label{fig_second_case}}}
%\caption{Simulation results}
%\label{fig_sim}
%\end{figure*}
%
% Note that often IEEE papers with subfigures do not employ subfigure
% captions (using the optional argument to \subfloat), but instead will
% reference/describe all of them (a), (b), etc., within the main caption.


% An example of a floating table. Note that, for IEEE style tables, the 
% \caption command should come BEFORE the table. Table text will default to
% \footnotesize as IEEE normally uses this smaller font for tables.
% The \label must come after \caption as always.
%
%\begin{table}[!t]
%% increase table row spacing, adjust to taste
%\renewcommand{\arraystretch}{1.3}
% if using array.sty, it might be a good idea to tweak the value of
% \extrarowheight as needed to properly center the text within the cells
%\caption{An Example of a Table}
%\label{table_example}
%\centering
%% Some packages, such as MDW tools, offer better commands for making tables
%% than the plain LaTeX2e tabular which is used here.
%\begin{tabular}{|c||c|}
%\hline
%One & Two\\
%\hline
%Three & Four\\
%\hline
%\end{tabular}
%\end{table}


% Note that IEEE does not put floats in the very first column - or typically
% anywhere on the first page for that matter. Also, in-text middle ("here")
% positioning is not used. Most IEEE journals/conferences use top floats
% exclusively. Note that, LaTeX2e, unlike IEEE journals/conferences, places
% footnotes above bottom floats. This can be corrected via the \fnbelowfloat
% command of the stfloats package.

\section{Conclusão}

O trabalho apresentou uma comparação entre os principais métodos de classifição diponíveis na literatura com o objetivo de determinar se um cidadão possui renda superior ou igual a 50 mil dólares anuais ou inferior a 50 mil dólares anuais. Os atributos disponibilizado em base dados reais e publicos foram extraídos do censo norte-americano de 1994.

Desconsiderando apenas o KNN, todos os outros métodos estudados apresentaram desempenho expressivos e muito semelhantes, variando menos de 2\% entre eles, destaca-se a Regressão Logística que apresentou as melhores medidas de desempenho, atigindo uma acurácia média de 86,52\% e uma F-medida média de 86,37\%. Destaca-se também, o desempenho de tempo do método Naive Bayes, com tempo médio de treinamento igual a 1,52 segundos, cerca de 20 vezes mais rápido que o segundo colocado e praticamente 250 vezes mais rápido que o último. O KNN apresentou desempenho satisfatório, porém 10\% inferior aos métodos mais avançados, sendo o método com 

Os métodos testados apresentaram ainda resultados ligeiramente superiores aos disponíveis na literatura, o que demonstra a capacidade dos métodos empregados em resolver a tarefa de classificação de renda. Trabalhos futuros incluem um maior refinamento dos parâmetros visando desempenhos ainda superiores e a paralelização dos algoritmos a fim de realizar o treinamento em menor tempo.



% conference papers do not normally have an appendix

% use section* for acknowledgement
\section*{Agradecimentos}
Os autores gostariam de agradecer ao docente da disciplina de Aprendizado de Máquina, Prof. Dr. Tiago Agostinho Almeida pela oportunidade de colocar em prática os conhecimentos adquiridos na disciplina e pela paciência para esclarecer as dúvidas.


% trigger a \newpage just before the given reference
% number - used to balance the columns on the last page
% adjust value as needed - may need to be readjusted if
% the document is modified later
%\IEEEtriggeratref{8}
% The "triggered" command can be changed if desired:
%\IEEEtriggercmd{\enlargethispage{-5in}}

% references section

% can use a bibliography generated by BibTeX as a .bbl file
% BibTeX documentation can be easily obtained at:
% http://www.ctan.org/tex-archive/biblio/bibtex/contrib/doc/
% The IEEEtran BibTeX style support page is at:
% http://www.michaelshell.org/tex/ieeetran/bibtex/
%\bibliographystyle{IEEEtran}
% argument is your BibTeX string definitions and bibliography database(s)
%\bibliography{IEEEabrv,../bib/paper}
%
% <OR> manually copy in the resultant .bbl file
% set second argument of \begin to the number of references
% (used to reserve space for the reference number labels box)
\begin{thebibliography}{1}

\bibitem{IEEEhowto:kopka}
H.~Kopka and P.~W. Daly, \emph{A Guide to \LaTeX}, 3rd~ed.\hskip 1em plus
  0.5em minus 0.4em\relax Harlow, England: Addison-Wesley, 1999.

\end{thebibliography}

% that's all folks
\end{document}