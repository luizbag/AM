\appendix[Tutorial para execução e reprodução de resultados]

Todos os métodos exibem informações como: número da partição, nome da tarefa executada, tempo de execução, acurácia na partição de treinamento e acurácia na partição de teste e ao final da execução exibe um agregado dos resultados obtidos, bem como medidas de desempenho adicional.

Os métodos Regressão Logística, Redes Neurais Artificiais, Máquinas de Vetores de Suporte, permitem o carregamento dos modelos previamente computados, a fim de se eliminar o tempo de treinamento. Os modelos que podem ser carregados são os modelos em que se obteve o melhor desempenho e portanto, os que estão documentados nas seções anteriores deste relatório.

Para a execução é necessário executar o arquivo principal projetoRenda.m

A primeira opção solicitada é a visualização dos dados em dimensão reduzida, digite a opção S para visualizar ou N para não visualizar. Em seguida, pode-se executar o cálculo das curvas de aprendizado, digite a opção S para executar as curvas de aprendizado ou N para não executar. Em seguida, pode-se executar os métodos de validação cruzada.

Caso escolha cálculo das curvas os parâmetros adicionais serão solicitados.

Caso escolha a opção de validação cruzada um menu solicitará qual método deverá ser executado. Digite 0 para executar todos os métodos ou 6 para sair.

Caso digite 1, os parâmetros adicionais do método do KNN serão solicitados, no caso o valor K, insira o valor K desejado e o método inciará sua execução para todas as partições. O valor de K selecionado para reproduzir os resultados é K = 51. Em seguida, o método KNN será executado para todas as partições.

Caso digite 2, os parâmetros adicionais do método da Regressão Logística serão soliticitados, um submenu solicitará qual a hipótese a ser utilizada, para se reproduzir os dados a selecionada dever ser 1 - Hipótese Linear, o próximo passo é escolher se deve carregar uma hipótese previamente calculada, digite S para carregar a hipótese computada para geração do relatório, ou N para computar novamente. Caso deseje-se computar novamente, deve-se escolher se o fator de regularização será utilizado, digite S para utilizar ou N para não utilizar, para se reproduzir os resultados digite S, se escolhido utilizar o fator de regularização, insira o valor do fator desejado, a fim de reproduzir os resultados, escolha o valor 100. Em seguida o método da Regressão Logística será executado.

Caso digite 3, os parâmetros adicionais do método de Redes Neurais Artificiais serão solicitados, um submenu solicitará se os \(\theta\) previamente computados devem ser recarregados, digite S para carregar ou N para não carregar. O próximo passo é selecionar se a quantidade de neurônios padrão será alterada. Digite S para alterar ou N para não alterar. Caso a opção escolhida seja S, digite a nova quantidade de neurônios na camada intermediária.
Em seguida o método das Redes Neurais Artificias será executado.

Caso digite 4, os parâmetros adicionais do método das máquinas de vetores de suporte serão solicitados, um submenu solicitará, se o modelo previamente computado do SVM deve ser carregado, digite S para carregar ou N para não carregar.

Caso digite 5, o método Naive Bayes será executado.

Após as execução das 10 partições, os resultados para cada partição serão exibidos, assim como as médias dos valores obtidos para o métodos. Após isso, os métodos: Regressão Logística, Redes Neurais Artifciais, Máquinas de Vetores de Suportes e Naive Bayes permite exportar os modelos computados, digite S para exportar (e sobrescrever caso um modelo previamente salvo exista) ou N para não exportar. Por fim, o menu principal será reexibido, caso deseje reexecutado


