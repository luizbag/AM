\section{Metodologia experimental}

Particionou-se a base de dados utilizando-se a metodologia de validação cruzada \emph{(k-fold cross-validation)}, visto que os dados não são sensíveis ao tempo. Utilizou-se 10 partições, sendo 9 delas para o treinamento e 1 para a validação, dessa forma os conjuntos de treinamento contém 40653 amostras e os conjuntos de teste 4517 escolhidas aleatoriamente.

Para avaliação do poder de classificação de cada método aplicou-se as medidas mais utilizadas, como acurácia, F-medida, precisão e revocação, contabilizando também o tempo de treinamento e teste de cada partição.

A fim de verificar a possibilidade de superajustamento ou subajustamento, gerou-se também os gráficos das curvas de aprendizado, realizando os treinamentos com partições incrementais, iniciando com 1 partição e finalizando com 9.

Para este relatório, utilizou-se normalização por padronização em todos os testes.

Apresentam-se aqui os parâmetros selecionados, a fim de possibilitar a reprodução dos resultados obtidos em cada método:

\subsection{KNN}

O KNN \emph{(K-vizinhos mais próximos)} é um método baseado em distâncias que consiste em selecionar os K vizinhos do conjunto de treinamento menos distantes da amostra de teste, e por distante entende-se o que apresenta a menor diferença entre os atributos.

O único parâmetro do KNN é o valor K, para o qual testou-se com os valores: 1, 3, 5, 7, 11, 21, 51. Para este relatório, selecionou-se K = 51.

\subsection{Regressão logística}

O método da regressão logística consiste em encontrar uma função \emph{(hipótese)} que classifique os atributos, minimizando o erro entre as amostras, através do ajuste dos coeficientes do polinômio \(\theta\).

Implementou-se 3 variações das hipóteses:

\begin{description}
\item[Hipótese Linear] \hfill \\ Atributos elevados a primeira potência;
\item[Hipótese Quadrática] \hfill \\ Atributos elevados a primeira e segunda potência;
\item[Hipótese Cúbica] \hfill \\ Atributos elevados a primeira, segunda e terceira potência;
\end{description}

A regressão logística ainda pode utilizar um parâmetro de regularização a fim de evitar o superajustamento ao conjunto de treinamento, balançeando a complexidade da hipótese.

Para seleção dos parâmetros testou-se, através de busca em grid, as 3 hipóteses, com parâmetro \(\lambda\) = 0, ou seja, sem regularização, e com a regularização variando de \(10^0\) a \(10^3\) com passo 1 na potência. Para este relatório, selecionou-se a hipótese linear com \(\lambda\) = 100.

\subsection{Redes Neurais Artificiais}

As Redes Neurais Artificias utilizadas foram os Perceptrons Multi-camadas que consistituem uma série de camadas massivamente conectadas de regressores logísticos, portanto, o método consiste em ajustar matrizes de coeficientes \(\theta\) a fim de minimizar o erro de classificação das amostras.

Entre os parâmetros a serem ajustados, existe a taxa de aprendizagem \(\alpha\), o número de camadas e o número de neurônios em cada camada.

Para este realatório selecionou-se \(\alpha\) = 0,06 e, como topologia da rede, a quantidade de atributos na camada de entrada, uma camada intermediária utilizando o teorema de CYBENKO e um neurônio na camada de saida para representar 1 como renda maior ou igual a 50 mil e 0 para menor que 50 mil.

Cybenko mostrou que uma única camada intermediária é suficiente para aproximar uniformemente qualquer função dado a sua caracterista de aproximador universal\cite{cybenko}. Dada a quantidade de atributos versus a quantidade de amostras usadas nos testes, não foi viável colocar mais de uma camada intermediária, pois isso afetaria muito o tempo de treinamento.

Variou-se os parametros de taxa de aprendizagem de 0,01 até 0,1 com incremento de 0,01. Tendo o melhor desempenho e velocidade de treinamento com 0,06. A quantidade de neurônios na camada intermediária foram ajustadas com 30, 50, 100, 150, 200 e 250, tendo o melhor desempenho utilizando 50 neurônios.

\subsection{SVM - Máquinas de vetores de suporte}

O SVM expande os atributos para um espaço de dimensão superior e encontra o hiperplano que fornece a maior margem entre os representantes de cada classe \cite{praticalSVM}. O SVM foi implementado utilizando-se a biblioteca LIBSVM \cite{libsvm}.

Os parâmetros incluem a seleção do kernel, dos coeficientes \emph{C}, que representa o parâmetro de custo (para os kernels linear, radial e polinomial) e \(\gamma\) (para os kernels radial e polinomial).

Testou-se o SVM com kernel linear, com \emph{C} com valores de \(10^{-4}\) a \(10^2\), com passo incremental de 1 na potência. Para os kernels radial e polinomial, testou-se, através de busca em grid, com \emph{C} variando de \(10^{-4}\) a \(10^2\) e \(\gamma\) variando de \(10^{-2}\) a \(10^2\), ambos com passo incremental de 1 na potência.

Para este relatório selecionou-se, C = 0.01 para o kernel linear e C = 1 e \(\gamma\) = 0.01 para o kernel radial.

\subsection{Naive Bayes}

O método Naive-Bayes se baseia nas probabilidades de ocorrência de cada classe e de cada atributo individualmente, sabendo a classe em que o mesmo se encontra. O método Naive-Bayes se baseia apenas nas probabilidades, portanto não possui parametros a serem ajustados.