\section{Resultados}

Para a normalização obteve-se resultados ligeiramente superiores (cerca de 1\%) utilizando-se normalização por padronização, portanto, esta foi a opção utlizada em todos os testes.

\subsection{KNN}

O único parâmetro do KNN é o valor K, para os resultados selecionou-se o valor K = 51, obtendo-se os resultados apresentados na Tabela \ref{table:resultadosKNN}:

\begin{table}[h]
\centering
\caption{Resultados para o KNN sendo K = 51}
\vspace{0.2cm}
\begin{tabular}{c|c|c|c|c|c}
Partição & Acurácia & F-medida & Precisão & Revocação & Tempo \\
\hline
1  & 0.74607 & 0.74607 & 0.74607 & 0.74607 & T \\     
2  & 0.75515 & 0.75515 & 0.75515 & 0.75515 & T \\     
3  & 0.74925 & 0.74925 & 0.74925 & 0.74925 & T \\     
4  & 0.75603 & 0.75603 & 0.75603 & 0.75603 & T \\     
5  & 0.76179 & 0.76179 & 0.76179 & 0.76179 & T \\     
6  & 0.74662 & 0.74662 & 0.74662 & 0.74662 & T \\     
7  & 0.75293 & 0.75293 & 0.75293 & 0.75293 & T \\     
8  & 0.75183 & 0.75183 & 0.75183 & 0.75183 & T \\     
9  & 0.75803 & 0.75803 & 0.75803 & 0.75803 & T \\     
10 & 0.74341 & 0.74341 & 0.74341 & 0.74341 & T \\
\hline
Média & 0.75211 & 0.75211 & 0.75211 & 0.75211 & T

\end{tabular} 
\label{table:resultadosKNN}
\end{table}

\subsection{Regressão logística}

Visando melhor desempenho de tempo, selecinou-se hipótese linear com um fator de regularização \(\lambda\)  = 1, obtendo-se os resultados apresentados na Tabela \ref{table:resultadosRL}


\begin{table}[h]
\centering
\caption{Resultados para a Regressão Logística sendo a hipótese linear e \(\lambda\) = 1}
\vspace{0.2cm}
\begin{tabular}{c|c|c|c|c|c}
Partição & Acurácia & F-medida & Precisão & Revocação & Tempo \\
\hline
1  & 0.85707 & 0.85605 & 0.85852 & 0.85503 & T \\      
2  & 0.8628  & 0.86187 & 0.8644  & 0.86086 & T \\      
3  & 0.85621 & 0.85517 & 0.85814 & 0.85411 & T \\      
4  & 0.85834 & 0.85739 & 0.8603  & 0.85639 & T \\      
5  & 0.86338 & 0.86256 & 0.86481 & 0.86167 & T \\      
6  & 0.8591  & 0.85808 & 0.86049 & 0.85705 & T \\      
7  & 0.86925 & 0.8682  & 0.8712  & 0.867   & T \\    
8  & 0.86938 & 0.86837 & 0.87098 & 0.86722 & T \\      
9  & 0.86246 & 0.86157 & 0.8641  & 0.86059 & T \\      
10 & 0.85825 & 0.85708 & 0.86031 & 0.85587 & T \\
\hline
Média & 0.86162 & 0.86063 & 0.86333 & 0.85958 & T 

\end{tabular} 
\label{table:resultadosRL}
\end{table}

\subsection{Redes Neurais Artificiais}

\subsection{Máquinas de vetores de suporte}

\subsection{Naive Bayes}
