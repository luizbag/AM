\section{Resultados}

\subsection{KNN}

O único parâmetro do KNN é o valor K, para os resultados selecionou-se o valor K = 51, obtendo-se os resultados apresentados na Tabela \ref{table:resultadosKNN}:

\begin{table}[h]
\centering
\caption{Resultados para o KNN sendo K = 51}
\vspace{0.2cm}
\begin{tabular}{c|c|c|c|c|c}
Partição & Acurácia & F-medida & Precisão & Revocação & Tempo \\
\hline
1  & 0.74607 & 0.74607 & 0.74607 & 0.74607 & 37.549 \\     
2  & 0.75515 & 0.75515 & 0.75515 & 0.75515 & 37.233 \\     
3  & 0.74925 & 0.74925 & 0.74925 & 0.74925 & 37.347 \\     
4  & 0.75603 & 0.75603 & 0.75603 & 0.75603 & 38.548 \\     
5  & 0.76179 & 0.76179 & 0.76179 & 0.76179 & 39.026 \\     
6  & 0.74662 & 0.74662 & 0.74662 & 0.74662 & 38.081 \\     
7  & 0.75293 & 0.75293 & 0.75293 & 0.75293 & 37.536 \\     
8  & 0.75183 & 0.75183 & 0.75183 & 0.75183 & 36.817 \\     
9  & 0.75803 & 0.75803 & 0.75803 & 0.75803 & 36.604 \\     
10 & 0.74341 & 0.74341 & 0.74341 & 0.74341 & 36.824 \\
\hline
Média & 0.75211 & 0.75211 & 0.75211 & 0.75211 & 37.557

\end{tabular} 
\label{table:resultadosKNN}
\end{table}

\subsection{Regressão logística}

Visando melhor desempenho de tempo, selecionou-se a hipótese linear com um fator de regularização \(\lambda\)  = 1, obtendo-se os resultados apresentados na Tabela \ref{table:resultadosRL}.


\begin{table}[h]
\centering
\caption{Resultados para a Regressão Logística sendo a hipótese linear e \(\lambda\) = 1}
\vspace{0.2cm}
\begin{tabular}{c|c|c|c|c|c}
Partição & Acurácia & F-medida & Precisão & Revocação & Tempo \\
\hline
1  & 0.85707 & 0.85605 & 0.85852 & 0.85503 & T \\      
2  & 0.86280 & 0.86187 & 0.86440 & 0.86086 & T \\      
3  & 0.85621 & 0.85517 & 0.85814 & 0.85411 & T \\      
4  & 0.85834 & 0.85739 & 0.86030 & 0.85639 & T \\      
5  & 0.86338 & 0.86256 & 0.86481 & 0.86167 & T \\      
6  & 0.85910 & 0.85808 & 0.86049 & 0.85705 & T \\      
7  & 0.86925 & 0.86820 & 0.87120 & 0.86700 & T \\    
8  & 0.86938 & 0.86837 & 0.87098 & 0.86722 & T \\      
9  & 0.86246 & 0.86157 & 0.86410 & 0.86059 & T \\      
10 & 0.85825 & 0.85708 & 0.86031 & 0.85587 & T \\
\hline
Média & 0.86162 & 0.86063 & 0.86333 & 0.85958 & T 

\end{tabular} 
\label{table:resultadosRL}
\end{table}

\subsection{Redes Neurais Artificiais}

\subsection{Máquinas de vetores de suporte}

\begin{table}[h]
\centering
\caption{Resultados para SVM com kernel linear e C = 0.01}
\vspace{0.2cm}
\begin{tabular}{c|c|c|c|c|c}
Partição & Acurácia & F-medida & Precisão & Revocação & Tempo \\
\hline
1  & 0.84725 & 0.84635 & 0.84835 & 0.84553 & 387.15 \\ 
2  & 0.85753 & 0.85665 & 0.85894 & 0.85575 & 371.86 \\
3  & 0.85985 & 0.85886 & 0.86134 & 0.85784 & 398.54 \\
4  & 0.85467 & 0.85379 & 0.85627 & 0.85291 & 396.04 \\
5  & 0.86017 & 0.85941 & 0.86129 & 0.85862 & 370.89 \\
6  & 0.85363 & 0.85271 & 0.85463 & 0.85186 & 371.20 \\
7  & 0.86295 & 0.86198 & 0.86457 & 0.86094 & 371.87 \\
8  & 0.86137 & 0.86044 & 0.86265 & 0.85948 & 373.39 \\
9  & 0.85854 & 0.85770 & 0.85991 & 0.85684 & 371.56 \\
10 & 0.85725 & 0.85617 & 0.85880 & 0.85508 & 371.22 \\
\hline
Média & 0.85732 & 0.85641 & 0.85867 & 0.85548 & 378.37 \\

\end{tabular} 
\label{table:resultadosSVMLinear}
\end{table}

\begin{table}[h]
\centering
\caption{Resultados para SVM com kernel radial, C = 1 e \(\gamma\) }
\vspace{0.2cm}
\begin{tabular}{c|c|c|c|c|c}
Partição & Acurácia & F-medida & Precisão & Revocação & Tempo \\
\hline
1  & 0.84725 & 0.84635 & 0.84835 & 0.84553 & 387.15 \\ 
2  & 0.85753 & 0.85665 & 0.85894 & 0.85575 & 371.86 \\
3  & 0.85985 & 0.85886 & 0.86134 & 0.85784 & 398.54 \\
4  & 0.85467 & 0.85379 & 0.85627 & 0.85291 & 396.04 \\
5  & 0.86017 & 0.85941 & 0.86129 & 0.85862 & 370.89 \\
6  & 0.85363 & 0.85271 & 0.85463 & 0.85186 & 371.20 \\
7  & 0.86295 & 0.86198 & 0.86457 & 0.86094 & 371.87 \\
8  & 0.86137 & 0.86044 & 0.86265 & 0.85948 & 373.39 \\
9  & 0.85854 & 0.85770 & 0.85991 & 0.85684 & 371.56 \\
10 & 0.85725 & 0.85617 & 0.85880 & 0.85508 & 371.22 \\
\hline
Média & 0.85732 & 0.85641 & 0.85867 & 0.85548 & 378.37 \\

\end{tabular} 
\label{table:resultadosSVMRadial}
\end{table}

\subsection{Naive Bayes}

